% Contenidos del capítulo.
% Las secciones presentadas son orientativas y no representan
% necesariamente la organización que debe tener este capítulo.

\section{Introducción}
Las tecnologías OpenBCI son una herramienta fundamental de la investigación de neurocientífica pioneros por 
democratizar el acceso a interfaces cerebro-computadora y aplicaciones biomédicas. 
OpenBCI es una plataforma de hardware y software abierto que permite la adquisición y análisis de señales bioeléctricas 
como EEG EMG ECG y EOG fundamentales para aplicaciones en neurociencia control por movimientos imaginarios aplicados principalmente en rehabilitación. 
En este trabajo vamos a centrarnos en los impulsos generados por los movimientos imaginarios, los cuales los registramos 
por electrodos que están conectados a las placas Cyton Ganglion y Daisy con alta precisión. 
Se monitoriza en tiempo real las señales bioeléctricas con gran precisión y se analizan a primera vista con la GUI que ofrece OpenBCI, 
más tarde para procesamiento avanzado se utiliza BrainFlow y OpenVibe.
Entre los desafíos y limitaciones se encuentran la calidad de las señales EEG que presentan niveles altos de ruido y sensibilidad a artefactos 
externos la complejidad en el procesamiento y clasificación de datos que requieren algoritmos avanzados y el rendimiento en velocidad y precisión que aún no iguala a sistemas invasivos.




\section{Motivación}


\section{Objetivos}


\section{Organización de la memoria}
